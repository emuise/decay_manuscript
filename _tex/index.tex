% Options for packages loaded elsewhere
\PassOptionsToPackage{unicode}{hyperref}
\PassOptionsToPackage{hyphens}{url}
\PassOptionsToPackage{dvipsnames,svgnames,x11names}{xcolor}
%
\documentclass[
]{agujournal2019}

\usepackage{amsmath,amssymb}
\usepackage{iftex}
\ifPDFTeX
  \usepackage[T1]{fontenc}
  \usepackage[utf8]{inputenc}
  \usepackage{textcomp} % provide euro and other symbols
\else % if luatex or xetex
  \usepackage{unicode-math}
  \defaultfontfeatures{Scale=MatchLowercase}
  \defaultfontfeatures[\rmfamily]{Ligatures=TeX,Scale=1}
\fi
\usepackage{lmodern}
\ifPDFTeX\else  
    % xetex/luatex font selection
\fi
% Use upquote if available, for straight quotes in verbatim environments
\IfFileExists{upquote.sty}{\usepackage{upquote}}{}
\IfFileExists{microtype.sty}{% use microtype if available
  \usepackage[]{microtype}
  \UseMicrotypeSet[protrusion]{basicmath} % disable protrusion for tt fonts
}{}
\makeatletter
\@ifundefined{KOMAClassName}{% if non-KOMA class
  \IfFileExists{parskip.sty}{%
    \usepackage{parskip}
  }{% else
    \setlength{\parindent}{0pt}
    \setlength{\parskip}{6pt plus 2pt minus 1pt}}
}{% if KOMA class
  \KOMAoptions{parskip=half}}
\makeatother
\usepackage{xcolor}
\setlength{\emergencystretch}{3em} % prevent overfull lines
\setcounter{secnumdepth}{5}
% Make \paragraph and \subparagraph free-standing
\makeatletter
\ifx\paragraph\undefined\else
  \let\oldparagraph\paragraph
  \renewcommand{\paragraph}{
    \@ifstar
      \xxxParagraphStar
      \xxxParagraphNoStar
  }
  \newcommand{\xxxParagraphStar}[1]{\oldparagraph*{#1}\mbox{}}
  \newcommand{\xxxParagraphNoStar}[1]{\oldparagraph{#1}\mbox{}}
\fi
\ifx\subparagraph\undefined\else
  \let\oldsubparagraph\subparagraph
  \renewcommand{\subparagraph}{
    \@ifstar
      \xxxSubParagraphStar
      \xxxSubParagraphNoStar
  }
  \newcommand{\xxxSubParagraphStar}[1]{\oldsubparagraph*{#1}\mbox{}}
  \newcommand{\xxxSubParagraphNoStar}[1]{\oldsubparagraph{#1}\mbox{}}
\fi
\makeatother


\providecommand{\tightlist}{%
  \setlength{\itemsep}{0pt}\setlength{\parskip}{0pt}}\usepackage{longtable,booktabs,array}
\usepackage{calc} % for calculating minipage widths
% Correct order of tables after \paragraph or \subparagraph
\usepackage{etoolbox}
\makeatletter
\patchcmd\longtable{\par}{\if@noskipsec\mbox{}\fi\par}{}{}
\makeatother
% Allow footnotes in longtable head/foot
\IfFileExists{footnotehyper.sty}{\usepackage{footnotehyper}}{\usepackage{footnote}}
\makesavenoteenv{longtable}
\usepackage{graphicx}
\makeatletter
\def\maxwidth{\ifdim\Gin@nat@width>\linewidth\linewidth\else\Gin@nat@width\fi}
\def\maxheight{\ifdim\Gin@nat@height>\textheight\textheight\else\Gin@nat@height\fi}
\makeatother
% Scale images if necessary, so that they will not overflow the page
% margins by default, and it is still possible to overwrite the defaults
% using explicit options in \includegraphics[width, height, ...]{}
\setkeys{Gin}{width=\maxwidth,height=\maxheight,keepaspectratio}
% Set default figure placement to htbp
\makeatletter
\def\fps@figure{htbp}
\makeatother
% definitions for citeproc citations
\NewDocumentCommand\citeproctext{}{}
\NewDocumentCommand\citeproc{mm}{%
  \begingroup\def\citeproctext{#2}\cite{#1}\endgroup}
\makeatletter
 % allow citations to break across lines
 \let\@cite@ofmt\@firstofone
 % avoid brackets around text for \cite:
 \def\@biblabel#1{}
 \def\@cite#1#2{{#1\if@tempswa , #2\fi}}
\makeatother
\newlength{\cslhangindent}
\setlength{\cslhangindent}{1.5em}
\newlength{\csllabelwidth}
\setlength{\csllabelwidth}{3em}
\newenvironment{CSLReferences}[2] % #1 hanging-indent, #2 entry-spacing
 {\begin{list}{}{%
  \setlength{\itemindent}{0pt}
  \setlength{\leftmargin}{0pt}
  \setlength{\parsep}{0pt}
  % turn on hanging indent if param 1 is 1
  \ifodd #1
   \setlength{\leftmargin}{\cslhangindent}
   \setlength{\itemindent}{-1\cslhangindent}
  \fi
  % set entry spacing
  \setlength{\itemsep}{#2\baselineskip}}}
 {\end{list}}
\usepackage{calc}
\newcommand{\CSLBlock}[1]{\hfill\break\parbox[t]{\linewidth}{\strut\ignorespaces#1\strut}}
\newcommand{\CSLLeftMargin}[1]{\parbox[t]{\csllabelwidth}{\strut#1\strut}}
\newcommand{\CSLRightInline}[1]{\parbox[t]{\linewidth - \csllabelwidth}{\strut#1\strut}}
\newcommand{\CSLIndent}[1]{\hspace{\cslhangindent}#1}

\usepackage{url} %this package should fix any errors with URLs in refs.
\usepackage{lineno}
\usepackage[inline]{trackchanges} %for better track changes. finalnew option will compile document with changes incorporated.
\usepackage{soul}
\linenumbers
\makeatletter
\@ifpackageloaded{caption}{}{\usepackage{caption}}
\AtBeginDocument{%
\ifdefined\contentsname
  \renewcommand*\contentsname{Table of contents}
\else
  \newcommand\contentsname{Table of contents}
\fi
\ifdefined\listfigurename
  \renewcommand*\listfigurename{List of Figures}
\else
  \newcommand\listfigurename{List of Figures}
\fi
\ifdefined\listtablename
  \renewcommand*\listtablename{List of Tables}
\else
  \newcommand\listtablename{List of Tables}
\fi
\ifdefined\figurename
  \renewcommand*\figurename{Figure}
\else
  \newcommand\figurename{Figure}
\fi
\ifdefined\tablename
  \renewcommand*\tablename{Table}
\else
  \newcommand\tablename{Table}
\fi
}
\@ifpackageloaded{float}{}{\usepackage{float}}
\floatstyle{ruled}
\@ifundefined{c@chapter}{\newfloat{codelisting}{h}{lop}}{\newfloat{codelisting}{h}{lop}[chapter]}
\floatname{codelisting}{Listing}
\newcommand*\listoflistings{\listof{codelisting}{List of Listings}}
\makeatother
\makeatletter
\makeatother
\makeatletter
\@ifpackageloaded{caption}{}{\usepackage{caption}}
\@ifpackageloaded{subcaption}{}{\usepackage{subcaption}}
\makeatother
\ifLuaTeX
  \usepackage{selnolig}  % disable illegal ligatures
\fi
\usepackage{bookmark}

\IfFileExists{xurl.sty}{\usepackage{xurl}}{} % add URL line breaks if available
\urlstyle{same} % disable monospaced font for URLs
\hypersetup{
  pdftitle={Spatially explicit mapping of ecological similarity across a large region using matching techniques.},
  pdfauthor={Evan Muise; Nicholas Coops},
  pdfkeywords={Protected areas, Matching methods, Coarsened exact
matching, Essential Biodiversity Variables},
  colorlinks=true,
  linkcolor={blue},
  filecolor={Maroon},
  citecolor={Blue},
  urlcolor={Blue},
  pdfcreator={LaTeX via pandoc}}

\journalname{Conservation Biology}

\draftfalse

\begin{document}
\title{Spatially explicit mapping of ecological similarity across a
large region using matching techniques.}

\authors{Evan Muise\affil{1}, Nicholas Coops\affil{1}}
\affiliation{1}{Department of Forest Resource Management, University of
British Columbia, Vancouver, 2424 Main Mall, British Columbia, Canada, }
\correspondingauthor{Evan Muise}{evanmuis@student.ubc.ca}


\begin{abstract}
ABSTRACT
\end{abstract}

\section*{Plain Language Summary}
Probably am not doing a plain language summary



\subsection{Introduction}\label{introduction}

A global biodiversity crisis is currently underway, driven by
anthropogenic changes (Dirzo \& Raven, 2003). Pressures such as climate
change, land-use changes, and invasive species, are leading to species
extinctions (Thomas et al., 2004; Urban, 2015) and the homogenization of
biological communities (McGill et al., 2015). The Kunming-Montreal
Global Biodiversity Framework (GBF) has been adopted as of December
2022, with the goal of restoring global biodiversity (\emph{Report of
the conference of the parties to the convention on biological diversity
on the second part of its fifteenth meeting}, 2023). Targets within this
framework include restoring 30\% of all degraded ecosystems, and
protecting 30\% of of Earth's terrestrial, inland water, and marine
areas (\emph{Report of the conference of the parties to the convention
on biological diversity on the second part of its fifteenth meeting},
2023). In the terrestrial environment, forest biomes have been shown to
harbour the largest amount of biodiversity (Cardinale et al., 2012;
Myers, 1988; Pimm \& Raven, 2000), and provide key ecosystem services
(Thompson et al., 2009). To provide these services, it is integral that
these forest ecosystems are in good ecological condition, as represented
by natural or near-natural levels of forest structure, forest function,
and forest composition (Marín et al., 2021).

While understanding forest condition is a key aspect of understanding
biodiversity and the provision of ecosystem services due to their
inherent linkages (Cardinale et al., 2012; Marín et al., 2021), it is
difficult to acquire suitable field-derived data across wide expanses of
land due to the extreme financial and temporal costs associated with
field campaigns. Remote sensing data, however, can provide a
time-efficient, cost-effective alternative to field data, providing
access to new spatially explicit and exhaustive datasets that can be
linked to ecological condition, with additional metrics being proposed
at a rapid pace (Pereira et al., 2013; Volker C. Radeloff et al., 2024;
Skidmore et al., 2021). Advances in lidar technologies and modelling
methods are allowing wall-to-wall estimates of forest stand structure to
be generated across entire countries (Becker et al., 2023; Matasci,
Hermosilla, Wulder, White, Coops, Hobart, Bolton, et al., 2018; Matasci,
Hermosilla, Wulder, White, Coops, Hobart, \& Zald, 2018), serving as a
finer scale indicator of ecosystem structure than the often previously
used landscape fragmentation metrics (Andrew et al., 2012). Productivity
metrics have been used as a proxy for ecosystem function for many years
(Pettorelli et al., 2005, 2018), with new Landsat derived datasets
providing integrative annual estimates of energy availability at a 30 m
spatial resolution(V. C. Radeloff et al., 2019; Volker C. Radeloff et
al., 2024; Razenkova et al., n.d.). Remote sensing is quickly providing
access to a plethora of datasets suitable monitoring the various facets
of biodiversity and ecological condition (Noss, 1990). Using these
datasets in conjunction with information on the location of known or
assumed natural forests can allow researchers to identify similar forest
stands or ecosystems, thus locating high quality forests across entire
jurisdictions, even when they are facing anthropogenic pressures.

However, identifying natural forests is not a simple task. There are
many definitions of what constitutes `naturalness' in forest
environments (Winter, 2012), and these difficulties are compounded by
shifting baseline syndrome (Pinnegar \& Engelhard, 2008) and a lack of
comprehensive historical data to generate reference states (Balaguer et
al., 2014). Other proposed methods to delineate baseline conditions
include protected areas (Arcese \& Sinclair, 1997), and empirical
estimates of the reference state, generated by modelling outcomes
(oftentimes species abundances and occurrence) in the absence of
anthropogenic disturbance (Nielsen et al., 2007).

In this study, we seek to identify natural forests of good ecological
condition at a 30 m spatial resolution across British Columbia, Canada.
We first identify reference states for ecosystems across the province as
protected areas with little-to-no anthropogenic disturbances. We then
use an expanded coarsened exact matching (CEM) approach to compare the
entire forested landscape of British Columbia to a reference state which
accounts for forest type as well as topographic and climatic
information. We expand the CEM technique to allow for the the spatially
explicit reconstruction of natural ecosystem similarity, even when a
perfect match is not available. We further examine how the coverage of
natural forest areas varies between protected and unprotected areas in
British Columbia. These methods will allow managers to identify regions
in need of restoration, or that could be suitable for protection, and
could be expanded to other regions.

\subsection{Methods}\label{sec-data-methods}

\subsubsection{Data}\label{data}

\paragraph{Natural forests}\label{natural-forests}

We defined natural forests as forests under formal protection with
little-to-no anthropogenic pressure. We obtained boundaries for the
protected areas of British Columbia using the \textbf{wdpar} R package
(Hanson, 2022 version 1.3.7), and filtered these protected areas larger
than 100 ha and in the IUCN categories Ia, Ib, II, and IV Muise et al.
(2022). We use the Canadian Human Footprint generated by Hirsh-Pearson
et al. (2022), a 300 m estimating of anthropogenic pressures across the
entirety of Canada, which accounts for pressures included in the global
human footprint (Venter et al., 2016), as well as pressures specific to
the Canadian environment. The Canadian Human Footprint considers built
environments, crop land, pasture land, human population density,
nighttime lights, railways, roads, navigable waterways, dams and
associated reservoirs, mining activity, oil and gas, and forestry
pressures (Hirsh-Pearson et al., 2022). We consider any pixel within our
protected area criteria with a Canadian Human Footprint score below four
as representative of natural ecosystems, which balances the tradeoff
between having a suitable number of high-quality pixels, and low
anthropogenic pressures being represented in the natural areas.

\paragraph{Forest Ecosystem Structure}\label{forest-ecosystem-structure}

Wall-to-wall, 30 m forest structure metrics (canopy height, canopy
cover, structural complexity, and aboveground biomass) were generated by
Matasci, Hermosilla, Wulder, White, Coops, Hobart, Bolton, et al. (2018)
for 2015 across the forested landscape of British Columbia. This data
was generated by using a random forest-kNN approach, imputing airborne
laser scanning derived forest structural attributes across the entirety
of Canada using Landsat-derived best-available-pixel (BAP) composites
(Hermosilla et al., 2016; White et al., 2014) and topographic
information (Matasci, Hermosilla, Wulder, White, Coops, Hobart, Bolton,
et al., 2018; Matasci, Hermosilla, Wulder, White, Coops, Hobart, \&
Zald, 2018). The BAP composites were derived by selecting optical
observations from the Landsat archive (including Landsat-5 Thematic
Mapper, Landsat-7 Enhanced Thematic Mapper Plus, and Landsat-8
Operational Land Imager imagery) over the course of the growing season,
considering atmospheric effects (haze, clouds, cloud shadows) and
distance from the desired composite date (in this case, August 31st).
Details on the pixel scoring method can be found in White et al. (2014).
The BAP composites were further refined by using a spectral trend
analysis on the normalized burn ratio to remove noise, and missing
pixels are infilled using temporally-interpolated values. This procedure
results in gap-free, surface-reflectance image composites (Hermosilla et
al., 2015).

\paragraph{Forest Ecosystem Function}\label{forest-ecosystem-function}

Landsat DHIs

\paragraph{Auxiliary Datasets}\label{auxiliary-datasets}

In our matching procedure (see Section ) we use two core datasets as our
covariates. Firstly, we use a 30 m digital elevation model and derived
slope dataset from the Advanced Spaceborne Thermal Emission and
Reflection Radiometer (ASTER) Version 2 GDEM product (Tachikawa et al.,
2011). We also match on four climate variables; mean annual
precipitation (MAP), mean annual temperature (MAT), mean warmest month
temperature (MWMT), and mean coldest month temperature (MCMT) calculated
from 1990-2020 climate normals using the ClimateNA software package at a
1 km spatial resolution, and downsampled to 30 m using cubic spline
resampling in the \textbf{terra} R package (Hijmans, 2024 version
1.7-71) in R (R Core Team, 2024 version 4.4.0).

\subsubsection{Matching}\label{sec-matching}

To identify natural forests across the province, we implement an
expanded coarsened exact matching (CEM) technique (Iacus et al., 2012).
This allows us to assess the similarity of all forested pixels in the
province to known natural forests, while controlling for covariates such
as climate and topography. CEM creates comparable groups of observation
among covariates by first coarsening the covariates into meaningful
bins. In our case, we coarsen all six of our covariates into 10
quantile-based bins. As an example, elevation observations in the 0-10th
percentile would be grouped into elevation bin one, the 10th-20th
percentile of observations into bin two, and so on. Following the
binning procedure, CEM performs exact matching on the coarsened
covariates, and discards any unmatched samples. We diverge from this
method by not dropping unmatched samples. Instead, we group our
unmatched strata with their nearest neighbour in quartile space, up to a
minimum of 30 samples, while minimizing the overall nearest neighbour
distance. This method combines the advantages of CEM (no bias checking,
simple computation Iacus et al., 2012), with the ability to generate
wall-to-wall comparisons of our treated cells to all other cells in the
province.

\subsubsection{Analysis}\label{analysis}

\subsection{Results}\label{results}

\subsection{Discussion}\label{discussion}

\subsection{Conclusion}\label{conclusion}

\subsection*{References}\label{references}
\addcontentsline{toc}{subsection}{References}

\phantomsection\label{refs}
\begin{CSLReferences}{1}{0}
\vspace{1em}

\bibitem[\citeproctext]{ref-andrew2012}
Andrew, M. E., Wulder, M. A., \& Coops, N. C. (2012). Identification of
{\emph{de facto}} protected areas in boreal canada. \emph{Biological
Conservation}, \emph{146}(1), 97--107.
\url{https://doi.org/10.1016/j.biocon.2011.11.029}

\bibitem[\citeproctext]{ref-arcese1997}
Arcese, P., \& Sinclair, A. R. E. (1997). The role of protected areas as
ecological baselines. \emph{The Journal of Wildlife Management},
\emph{61}(3), 587--602. \url{https://doi.org/10.2307/3802167}

\bibitem[\citeproctext]{ref-balaguer2014}
Balaguer, L., Escudero, A., Martín-Duque, J. F., Mola, I., \& Aronson,
J. (2014). The historical reference in restoration ecology: Re-defining
a cornerstone concept. \emph{Biological Conservation}, \emph{176},
12--20. \url{https://doi.org/10.1016/j.biocon.2014.05.007}

\bibitem[\citeproctext]{ref-becker2023}
Becker, A., Russo, S., Puliti, S., Lang, N., Schindler, K., \& Wegner,
J. D. (2023). Country-wide retrieval of forest structure from optical
and SAR satellite imagery with deep ensembles. \emph{ISPRS Journal of
Photogrammetry and Remote Sensing}, \emph{195}, 269--286.
\url{https://doi.org/10.1016/j.isprsjprs.2022.11.011}

\bibitem[\citeproctext]{ref-bolton2019}
Bolton, D. K., Coops, N. C., Hermosilla, T., Wulder, M. A., White, J.
C., \& Ferster, C. J. (2019). Uncovering regional variability in
disturbance trends between parks and greater park ecosystems across
canada (1985{\textendash}2015). \emph{Scientific Reports}, \emph{9}(1),
1323. \url{https://doi.org/10.1038/s41598-018-37265-4}

\bibitem[\citeproctext]{ref-cardinale2012}
Cardinale, B. J., Duffy, J. E., Gonzalez, A., Hooper, D. U., Perrings,
C., Venail, P., et al. (2012). Biodiversity loss and its impact on
humanity. \emph{Nature}, \emph{486}(7401), 59--67.
\url{https://doi.org/10.1038/nature11148}

\bibitem[\citeproctext]{ref-dirzo2003}
Dirzo, R., \& Raven, P. H. (2003). Global State of Biodiversity and
Loss. \emph{Annual Review of Environment and Resources},
\emph{28}(Volume 28, 2003), 137--167.
\url{https://doi.org/10.1146/annurev.energy.28.050302.105532}

\bibitem[\citeproctext]{ref-wdpar2022a}
Hanson, J. O. (2022). Wdpar: Interface to the world database on
protected areas. \emph{Journal of Open Source Software}, \emph{7}, 4594.
\url{https://doi.org/10.21105/joss.04594}

\bibitem[\citeproctext]{ref-hermosilla2015}
Hermosilla, T., Wulder, M. A., White, J. C., Coops, N. C., \& Hobart, G.
W. (2015). Regional detection, characterization, and attribution of
annual forest change from 1984 to 2012 using landsat-derived time-series
metrics. \emph{Remote Sensing of Environment}, \emph{170}, 121132.
\url{https://doi.org/10.1016/j.rse.2015.09.004}

\bibitem[\citeproctext]{ref-hermosilla2016}
Hermosilla, T., Wulder, M. A., White, J. C., Coops, N. C., Hobart, G.
W., \& Campbell, L. B. (2016). Mass data processing of time series
landsat imagery: Pixels to data products for forest monitoring.
\emph{International Journal of Digital Earth}, \emph{9}(11), 10351054.
\url{https://doi.org/10.1080/17538947.2016.1187673}

\bibitem[\citeproctext]{ref-R-terra}
Hijmans, R. J. (2024). \emph{Terra: Spatial data analysis}. Retrieved
from \url{https://rspatial.org/}

\bibitem[\citeproctext]{ref-hirsh-pearson2022}
Hirsh-Pearson, K., Johnson, C. J., Schuster, R., Wheate, R. D., \&
Venter, O. (2022). Canada{'}s human footprint reveals large intact areas
juxtaposed against areas under immense anthropogenic pressure.
\emph{FACETS}, \emph{7}, 398--419.
\url{https://doi.org/10.1139/facets-2021-0063}

\bibitem[\citeproctext]{ref-iacus2012}
Iacus, S. M., King, G., \& Porro, G. (2012). Causal Inference without
Balance Checking: Coarsened Exact Matching. \emph{Political Analysis},
\emph{20}(1), 1--24. \url{https://doi.org/10.1093/pan/mpr013}

\bibitem[\citeproctext]{ref-maruxedn2021}
Marín, A. I., Abdul Malak, D., Bastrup-Birk, A., Chirici, G., Barbati,
A., \& Kleeschulte, S. (2021). Mapping forest condition in europe:
Methodological developments in support to forest biodiversity
assessments. \emph{Ecological Indicators}, \emph{128}, 107839.
\url{https://doi.org/10.1016/j.ecolind.2021.107839}

\bibitem[\citeproctext]{ref-matasci2018b}
Matasci, G., Hermosilla, T., Wulder, M. A., White, J. C., Coops, N. C.,
Hobart, G. W., \& Zald, H. S. J. (2018). Large-area mapping of Canadian
boreal forest cover, height, biomass and other structural attributes
using Landsat composites and lidar plots. \emph{Remote Sensing of
Environment}, \emph{209}, 90--106.
\url{https://doi.org/10.1016/j.rse.2017.12.020}

\bibitem[\citeproctext]{ref-matasci2018}
Matasci, G., Hermosilla, T., Wulder, M. A., White, J. C., Coops, N. C.,
Hobart, G. W., Bolton, D. K., et al. (2018). Three decades of forest
structural dynamics over canada's forested ecosystems using landsat
time-series and lidar plots. \emph{Remote Sensing of Environment},
\emph{216}, 697714. \url{https://doi.org/10.1016/j.rse.2018.07.024}

\bibitem[\citeproctext]{ref-mcgill2015}
McGill, B. J., Dornelas, M., Gotelli, N. J., \& Magurran, A. E. (2015).
Fifteen forms of biodiversity trend in the Anthropocene. \emph{Trends in
Ecology \& Evolution}, \emph{30}(2), 104--113.
\url{https://doi.org/10.1016/j.tree.2014.11.006}

\bibitem[\citeproctext]{ref-muise2022}
Muise, E. R., Coops, N. C., Hermosilla, T., \& Ban, S. S. (2022).
Assessing representation of remote sensing derived forest structure and
land cover across a network of protected areas. \emph{Ecological
Applications}, \emph{32}(5), e2603.
\url{https://doi.org/10.1002/eap.2603}

\bibitem[\citeproctext]{ref-myers1988}
Myers, N. (1988). Threatened biotas: {"}Hot spots{"} in tropical
forests. \emph{Environmentalist}, \emph{8}(3), 187--208.
\url{https://doi.org/10.1007/BF02240252}

\bibitem[\citeproctext]{ref-nielsen2007}
Nielsen, S. E., Bayne, E. M., Schieck, J., Herbers, J., \& Boutin, S.
(2007). A new method to estimate species and biodiversity intactness
using empirically derived reference conditions. \emph{Biological
Conservation}, \emph{137}(3), 403--414.
\url{https://doi.org/10.1016/j.biocon.2007.02.024}

\bibitem[\citeproctext]{ref-noss1990}
Noss, R. F. (1990). Indicators for Monitoring Biodiversity: A
Hierarchical Approach. \emph{Conservation Biology}, \emph{4}(4),
355--364. \url{https://doi.org/10.1111/j.1523-1739.1990.tb00309.x}

\bibitem[\citeproctext]{ref-pereira2013}
Pereira, H. M., Ferrier, S., Walters, M., Geller, G. N., Jongman, R. H.
G., Scholes, R. J., et al. (2013). Essential Biodiversity Variables.
\emph{Science}, \emph{339}(6117), 277--278.
\url{https://doi.org/10.1126/science.1229931}

\bibitem[\citeproctext]{ref-pettorelli2005}
Pettorelli, N., Vik, J. O., Mysterud, A., Gaillard, J.-M., Tucker, C.
J., \& Stenseth, N. Chr. (2005). Using the satellite-derived NDVI to
assess ecological responses to environmental change. \emph{Trends in
Ecology \& Evolution}, \emph{20}(9), 503--510.
\url{https://doi.org/10.1016/j.tree.2005.05.011}

\bibitem[\citeproctext]{ref-pettorelli2018}
Pettorelli, N., Schulte to Bühne, H., Tulloch, A., Dubois, G.,
Macinnis-Ng, C., Queirós, A. M., et al. (2018). Satellite remote sensing
of ecosystem functions: opportunities, challenges and way forward.
\emph{Remote Sensing in Ecology and Conservation}, \emph{4}(2), 71--93.
\url{https://doi.org/10.1002/rse2.59}

\bibitem[\citeproctext]{ref-pimm2000}
Pimm, S. L., \& Raven, P. (2000). Extinction by numbers. \emph{Nature},
\emph{403}(6772), 843--845. \url{https://doi.org/10.1038/35002708}

\bibitem[\citeproctext]{ref-pinnegar2008}
Pinnegar, J. K., \& Engelhard, G. H. (2008). The {`}shifting baseline{'}
phenomenon: a global perspective. \emph{Reviews in Fish Biology and
Fisheries}, \emph{18}(1), 1--16.
\url{https://doi.org/10.1007/s11160-007-9058-6}

\bibitem[\citeproctext]{ref-R-base}
R Core Team. (2024). \emph{R: A language and environment for statistical
computing}. Vienna, Austria: R Foundation for Statistical Computing.
Retrieved from \url{https://www.R-project.org/}

\bibitem[\citeproctext]{ref-radeloff2019}
Radeloff, V. C., Dubinin, M., Coops, N. C., Allen, A. M., Brooks, T. M.,
Clayton, M. K., et al. (2019). The Dynamic Habitat Indices (DHIs) from
MODIS and global biodiversity. \emph{Remote Sensing of Environment},
\emph{222}, 204--214. \url{https://doi.org/10.1016/j.rse.2018.12.009}

\bibitem[\citeproctext]{ref-radeloff2024}
Radeloff, Volker C., Roy, D. P., Wulder, M. A., Anderson, M., Cook, B.,
Crawford, C. J., et al. (2024). Need and vision for global
medium-resolution landsat and sentinel-2 data products. \emph{Remote
Sensing of Environment}, \emph{300}, 113918.
\url{https://doi.org/10.1016/j.rse.2023.113918}

\bibitem[\citeproctext]{ref-razenkova}
Razenkova, E., Lewińska, K. E., Yin, H., Farwell, L. S., Pidgeon, A. M.,
Hostert, P., et al. (n.d.). Medium-resolution dynamic habitat indices
from landsat satellite imagery. \emph{Remote Sensing of Environment}.

\bibitem[\citeproctext]{ref-reporto2023}
\emph{Report of the conference of the parties to the convention on
biological diversity on the second part of its fifteenth meeting}.
(2023).

\bibitem[\citeproctext]{ref-skidmore2021}
Skidmore, A. K., Coops, N. C., Neinavaz, E., Ali, A., Schaepman, M. E.,
Paganini, M., et al. (2021). Priority list of biodiversity metrics to
observe from space. \emph{Nature Ecology \& Evolution}.
\url{https://doi.org/10.1038/s41559-021-01451-x}

\bibitem[\citeproctext]{ref-tachikawa2011}
Tachikawa, T., Kaku, M., Iwasaki, A., Gesch, D. B., Oimoen, M. J.,
Zhang, Z., et al. (2011). \emph{ASTER global digital elevation model
version 2 - summary of validation results} (p. 27). Retrieved from
\url{http://pubs.er.usgs.gov/publication/70005960}

\bibitem[\citeproctext]{ref-thomas2004}
Thomas, C. D., Cameron, A., Green, R. E., Bakkenes, M., Beaumont, L. J.,
Collingham, Y. C., et al. (2004). Extinction risk from climate change.
\emph{Nature}, \emph{427}, 5.

\bibitem[\citeproctext]{ref-thompson2009}
Thompson, I. D., Mackey, B., McNulty, S., \& Mosseler, A. (2009).
\emph{Forest resilience, biodiversity, and climate change: a synthesis
of the biodiversity / resiliende / stability relationship in forest
ecosystems}. Montreal: Secretariat of the Convention on Biological
Diversity.

\bibitem[\citeproctext]{ref-urban2015}
Urban, M. C. (2015). Accelerating extinction risk from climate change.
\emph{Science}, \emph{348}(6234), 571--573.
\url{https://doi.org/10.1126/science.aaa4984}

\bibitem[\citeproctext]{ref-venter2016}
Venter, O., Sanderson, E. W., Magrach, A., Allan, J. R., Beher, J.,
Jones, K. R., et al. (2016). Sixteen years of change in the global
terrestrial human footprint and implications for biodiversity
conservation. \emph{Nature Communications}, \emph{7}(1), 12558.
\url{https://doi.org/10.1038/ncomms12558}

\bibitem[\citeproctext]{ref-white2014}
White, Joanne. C., Wulder, M. A., Hobart, G. W., Luther, J. E.,
Hermosilla, T., Griffiths, P., et al. (2014). Pixel-based image
compositing for large-area dense time series applications and science.
\emph{Canadian Journal of Remote Sensing}, \emph{40}(3), 192212.
\url{https://doi.org/10.1080/07038992.2014.945827}

\bibitem[\citeproctext]{ref-winter2012}
Winter, S. (2012). Forest naturalness assessment as a component of
biodiversity monitoring and conservation management. \emph{Forestry},
\emph{85}(2), 293--304. \url{https://doi.org/10.1093/forestry/cps004}

\end{CSLReferences}



\end{document}
